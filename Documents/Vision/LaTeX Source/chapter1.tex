
\section{Introduction}
\setcounter{section}{1}
\subsection{Objectif du document}
Le document de vision définit les objectifs principaux et la vision d’un projet. Il y sera décrit le problème et la solution proposée.
\subsection{Portée}
Il s’agit d’un projet visant à mettre en relation des étudiants et le corps universitaire afin de faciliter l’entraide entre eux.

\subsection{Définitions, Acronymes, et Abréviations}
\begin{itemize}
	\item Utilisateur : n’importe quelle personne physique utilisant le site après s’être identifié.
	\item Visiteur : n’importe quelle personne physique utilisant le site sans s’être identifié.
	\item Modérateur : utilisateur pouvant modérer (ajout/suppression de posts).
	\item Administrateur : utilisateur ayant tous les pouvoirs sur le site.

\end{itemize}

\noindent Modérateurs \& Administrateurs sont aussi des utilisateurs.

\subsection{Vue générale du document}
Ce document présente le positionnement commercial du produit, avec une définition du problème ainsi qu’une solution potentielle. Les intervenants et utilisateurs y seront présentés. Une vue d’ensemble du produit et de ses caractéristiques y sera aussi présentée. Finalement, une présentation des contraintes du produit et des tolérances d’utilisation seront données.



\section{Positionnement}
\subsection{Opportunité commerciale}
Mettre en relation facilement et de manière efficace les étudiants et le corps universitaire, susceptibles de répondre à un problème particulier.


\subsection{Position du problème}
\begin{itemize}
	\item Problème: Lorsqu'un étudiant rencontre une difficulté et ne trouve pas d'aide en ligne, ses seuls ressorts sont d’autres étudiants à l'aise dans le domaine touché, ou les enseignants. Mais, ces derniers ont souvent des disponibilités réduites.
	\item Affecte: Les étudiants et le corps universitaire.
	\item L'impact étant: Les étudiants n’ont pas de moyen efficace à leur disposition pour répondre à leur problème.
	\item Une bonne solution serait: Avoir une plateforme mettant en lien les étudiants du monde entier ainsi que les différents corps universitaires.
\end{itemize}

\subsection{Position du produit}
\begin{itemize}
	\item Pour: les étudiants
	\item Qui: ne trouvent pas de solution à leur problème
	\item UniHub: est une plateforme web de type forum
	\item Qui: permet l’entraide des étudiants
	\item Contrairement à: des solutions existantes comme Moodle ou Chamilo.
	\item Notre produit: est plus épuré, complet, et est ouvert à plus de monde.
\end{itemize}



\section{Description des intervenants et des utilisateurs}
\subsection{Taille du marché}
\indent Quelle est la réputation de notre entreprise dans ces marchés : inexistante.\\
\noindent Que souhaiteriez-vous qu’elle soit ? Il faudrait que notre réputation grandisse.\\
\noindent Comment le produit ou service envisagé contribute-t-ils à atteindre vos objectifs? Il proposerait une solution novatrice.\\

\newpage
\subsection{Les intervenants}
\begin{longtable}{| N M{70pt} | ^ M{110pt} | ^ M{142pt} |}
	\hline
	\rowstyle{\bfseries} Nom & Description & Responsabilitées \tabularnewline
	\hline
	\endhead
	\hline
	Etudiants & Ont besoin d'aide ou peuvent aider & Aider les autres
	\tabularnewline
	\hline
	Professeurs ou Assistants & Peuvent guider les étudiants à travers leur recherches & Faciliter les recherches des étudiants \tabularnewline
	\hline
	
	
\end{longtable}

\subsection{Les utilisateurs}
\begin{longtable}{| N M{70pt} | ^ M{110pt} | ^ M{142pt} |}
	\hline
	\rowstyle{\bfseries}  Nom & Description & Responsabilitées \tabularnewline
	\hline
	\endhead
	\hline
	Utilisateurs & C’est eux qui posent les questions et qui peuvent y répondre &
	\begin{itemize}
		\item Posent des questions
		\item Répondent aux questions
		\item Partagent des informations administratives
	\end{itemize}
	\tabularnewline
	\hline
	Visiteurs & Ils peuvent lire les questions et y répondre & \begin{itemize}
		\item Se renseigner en parcourant les threads existant
	\end{itemize} \tabularnewline
	\hline
	
	Modérateur ou Administrateur & Maintient l'ordre sur le forum &
	\begin{itemize}
		\item Fait respecter les règles
		\item Partage des informations
	\end{itemize} \tabularnewline
	\hline
\end{longtable}

\newpage

\subsection{Environnement utilisateur}

Application web sur laquelle les utilisateurs peuvent poser et répondre aux questions.

\subsection{Profiles des intervenants}
\subsubsection{Etudiants}

\begin{center}

\begin{tabular}{| >{\bfseries} M{85pt} | M{250pt} |}
	\hline
	Description & Un étudiant au niveau universitaire \tabularnewline
	\hline
	Type & Un utilisateur "banal" \tabularnewline
	\hline
	Respons. &
	\begin{itemize}
		\item Poser des questions
		\item Répondre aux questions
	\end{itemize} \tabularnewline
	\hline
	Critère de succès &
	\begin{itemize}
		\item Obtenir une réponse
		\item Aider les autres en donnant une réponse
	\end{itemize} \tabularnewline
	\hline
	Implication & Utilisateur principal \tabularnewline
	\hline

\end{tabular}

\end{center}

\subsubsection{Professeurs et Assistants}

\begin{center}

\begin{tabular}{| >{\bfseries} M{85pt} | M{250pt} |}
	\hline
	Description & Un étudiant qui à le statut d'assistant ou un professeur \tabularnewline
	\hline
	Type & Un utilisateur "banal" \tabularnewline
	\hline
	Respons. &
	\begin{itemize}
		\item Répondre aux questions
	\end{itemize} \tabularnewline
	\hline
	Critère de succès &
	\begin{itemize}
		\item Aider les étudiants en donnant une réponse
	\end{itemize} \tabularnewline
	\hline
	Implication & Utilisateur de l'application web \tabularnewline
	\hline

\end{tabular}

\end{center}


\subsection{Profiles des utilisateurs}
\subsubsection{Utilisateur}

\begin{center}

\begin{tabular}{| >{\bfseries} M{85pt} | M{250pt} |}
	\hline
	Représentant & Un étudiant ou une personne du corps universitaire \tabularnewline
	\hline
	Description & Toute personne physique inscrite sur le site \tabularnewline
	\hline
	Type & Etudiant ou corps universitaire \tabularnewline
	\hline
	Respons. &
	\begin{itemize}
		\item Répondre aux questions
	\end{itemize} \tabularnewline
	\hline
	Critère de succès &
	\begin{itemize}
		\item Répond correctement à une question
		\item Poste une annonce concernant les étudiants
		\item Poste une question d'intérêt général
	\end{itemize} \tabularnewline
	\hline
	Implication & Répondre aux questions \tabularnewline
	\hline
	Livrables & Un commentaire ou un poste \tabularnewline
	\hline
\end{tabular}

\end{center}

\subsubsection{Visiteur}

\begin{center}

\begin{tabular}{| >{\bfseries} M{85pt} | M{250pt} |}
	\hline
	Représentant & N'importe quelle personne physique \tabularnewline
	\hline
	Description & Toute personne physique non inscrite sur le site \tabularnewline
	\hline
	Type & Etudiant ou corps universitaire ou autre \tabularnewline
	\hline
	Respons. &
	 \tabularnewline
	\hline
	Critère de succès &
	\tabularnewline
	\hline
	Implication & 
	\tabularnewline
	\hline
	Livrables &  
	\tabularnewline
	\hline
\end{tabular}

\end{center}

\subsubsection{Modérateur}

\begin{center}
\begin{tabular}{| >{\bfseries} M{85pt} | M{250pt} |}
	\hline
	Représentant & Un étudiant ou une personne du corps universitaire \tabularnewline
	\hline
	Description & Toute personne physique inscrite sur le site ayant le statut de modérateur (accordée par l'administrateur) \tabularnewline
	\hline
	Type & Etudiant ou corps universitaire \tabularnewline
	\hline
	Respons. &
	\begin{itemize}
		\item Assurer le maintient de l'ordre sur le site
	\end{itemize} \tabularnewline
	\hline
	Critère de succès &
	\begin{itemize}
	    \item Modération du site (pas de débordement de comportement)
		\item Répond correctement à une question
		\item Poste une annonce concernant les étudiants
		\item Poste une question d'intérêt général
	\end{itemize} \tabularnewline
	\hline
	Implication & Modérer \tabularnewline
	\hline
	Livrables &  
	\tabularnewline
	\hline
\end{tabular}

\end{center}

\subsubsection{Administrateur}

\begin{center}

\begin{tabular}{| >{\bfseries} M{85pt} | M{250pt} |}
	\hline
	Représentant & Un étudiant ou une personne du corps universitaire \tabularnewline
	\hline
	Description & Toute personne physique inscrite sur le site ayant le statut d'administrateur \tabularnewline
	\hline
	Type & Etudiant ou corps universitaire \tabularnewline
	\hline
	Respons. &
	\begin{itemize}
		\item Gérer le site
	\end{itemize} \tabularnewline
	\hline
	Critère de succès &
	\begin{itemize}
		\item Administrer le site
	\end{itemize} \tabularnewline
	\hline
	Implication &  
	\tabularnewline
	\hline
	Livrables &  
	\tabularnewline
	\hline
\end{tabular}

\end{center}

\newpage

\subsection{Besoins clés des intervenants et utilisateurs}
\begin{longtable}{| N M{45pt} | ^ M{45pt} | ^ M{65pt} | ^ M{73pt} | ^ M{70pt} |}
	\hline
	\rowstyle{\bfseries} Besoin métier & Priorité & Concerne & Solution actuelle & Solution proposée \tabularnewline
	\hline
	\endhead
	\hline
	Obtenir une réponse à une quesiton & Elevée & Etudiants & Poser une question sur un moteur de recherche ou demander à un assitant & Chercher la réponse sur une plateforme qui centralise les besoins d'un étudiant \tabularnewline
	\hline
	Répondre à une question & Elevée & Etudiants, Assistants, Professeurs & Répondre sur les forums existants & Répondre sur notre plateforme
	\tabularnewline
	\hline
	Publier des annonces d'intérêt général concernant un cours & Elevée & Tous les utilisateurs & None & Créer une annonce sur notre plateforme
	\tabularnewline
	\hline
\end{longtable}

\subsection{Alternatives et concurrence}
\subsubsection{StackOverflow}

\subsubsection{Moodle, Chamilo}

\subsubsection{Quora}

\section{Vue d'ensemble du produit}
\subsection{Perspective du produit}

Notre produit est un stand alone. Il agit avec une base de données et le système des utilisateurs en temps que web-app.

\subsection{Résumé des caractéristiques}
\begin{longtable}{| N M{168pt} | ^ M{167pt} |}
	\hline
	\rowstyle{\bfseries} Avantage pour l'utilisateur & Caractéristiques correspondantes \tabularnewline
	\hline
	\endhead
	\hline
	Les profils qui ont le plus de réponses acceptées sont repérés & Un badge est affiché à côté de leur profile \tabularnewline
	\hline
	Les profils (ou posts) de spam sont repérés & Suspension des droits utilisateurs de manière temporaire ou définitive \tabularnewline
	\hline
	Auto-complétion de la recherche utilisateur en fonction de ce qu'il écrit & Bot \tabularnewline
	\hline
\end{longtable}

\subsection{Hypothèses}

Si personne ne répond, l'étudiant n'est pas aidé. 

\subsection{Coût et politique de prix}
Site à but non lucratif
\subsection{Licenses et installation}
Pas de license.

\section{Caractéristiques essentielles du produit}
Le site permet aux étudiants de différentes universités de s’inscrire, poser des questions sur des sujets particuliers et aider d’autres étudiants. Un étudiant ne peut poser une question qu’au sein de son université, mais il peut répondre aux questions posées dans les autres universités. Chaque étudiant peut upvoter / downvoter une réponse si celle-ci l’a aidé ou non ; et la personne ayant posé la question peut désigner la meilleure réponse qu’il a reçu pour que celle-ci soit mise en évidence.\\

\noindent Les user-stories détaillées sont disponibles sur notre repo GitHub dans le dossier documents.


\section{Contraintes sur le produit}

Avoir suffisamment d’utilisateurs pour le bon fonctionnement du site. La question posée doit avoir un lien avec les tags utilisés lors de la rédaction de celle-ci. Les réponses ne doivent pas être hors-sujet.

\newpage

\section{Tolérances de qualité non fonctionnelles}
\begin{itemize}
	\item Fluidité: le site doit avoir un temps de réponse cours
	\item Pratique: les fonctionnalités doivent être faciles à trouver pour l'utilisateur
	\item Accessilibité: le site doit être disponible en tout temps
	\item Sécurisé: le site ne doit pas exposer de failles de sécurités majeures
\end{itemize}



\section{Priorité des mutuelles des caractéristiques}
\begin{itemize}
		\item Les plus prioritaires: créer un compte, se loger et pouvoir créer ou répondre à un post
		\item Les moins prioritaires: le système de badge et d'upvote, le bot pour la complétion automatique des recherches utilisateurs
\end{itemize}



\section{Autres exigences sur le produit}

Le produit doit fonctionner sur tous les navigateurs web principaux (Chrome, Firefox etc).

\subsection{Standards applicables}
Fluidité et fiabilité.

\subsection{Besoins systèmes}

Une connexion internet et un navigateur internet.


\subsection{Performance}

Doit accepter des multiples connexions simultannées

\subsection{Exigences liées à l’environnement}

Site et base de données tout le temps disponible.

\section{Exigence de documentation}
\subsection{Manuel utilisateur}

\subsection{Aide en ligne}

\subsection{Guides d’installation et de configuration, fichier readme}

\section{Packaging, labelling, copyright}